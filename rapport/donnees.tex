\chapter{Synthèse des Données}

\section{Résumé des données}

\begin{table}[h]
\centering
\begin{tabularx}{\textwidth}{|X|X|X|X|}
\hline
\textbf{Appartement} & \textbf{Locataire} & \textbf{Contrat} & \textbf{Bâtiment}\\
\hline
Type d'appartement & Nom du locataire & Période & Numéro du Bâtiment \\
\hline
Nombre de locataires & Prénom du locataire & Montant du Loyer & Nombre d'appartements \\
\hline
Numéro du bâtiment & Contact du locatiare & & Quantité de déchet \\
\hline
Quantité de déchet & & & Consommation en eau et électricité \\
\hline
Consommation en eau et electricité & & & \\
\hline
\end{tabularx}
\caption{Tableau des données retenues pour la modélisation}
\end{table}

\section{Explication concernant le choix des données}

Bien qu'une grande partie de ces données répondent directements à la problématique énoncée plus tôt. Nous avons pris la liberté d'en ajouter quelques une qui vous permettront une meillleure gestion de cet ecoquartier, notamment :
\begin{itemize}
\item \textbf{Le nom, prénom et contact du locataire :} Dans l'optique promouvoir chez ses locataires une gestion plus durable de leurs déchets, notre client nous a fait part de son envie d'organiser des événements les impliquant directement dans ce processus, aussi avons-nous pensé utile de permettre de trouver facilement un moyen de les joindre.
\item \textbf{La consommation et les déchets par appartement :} Notre client souhaitait avoir cette valeur pour chaque bâtiment. Cette donnée supplémentaire offrira toujours cette possibilité, en plus de pouvoir cibler des appartements plus énergivores afin de réfléchir à une solution pour diminuer leur impact sur la consommation globale d'un bâtiment.
\end{itemize}
